\documentclass[12pt]{extarticle}
\usepackage[margin=0.3in, top=1in, bottom=1in]{geometry}
\usepackage[utf8]{inputenc}
\usepackage{amsmath,amssymb,amsthm,mathtools}
\usepackage{thmtools, thm-restate}
\usepackage{changepage}
\usepackage[T1]{fontenc}
\usepackage{eulervm,eucal,eufrak}% the math fonts used in concrete mathematics
\usepackage{bbold}
\usepackage{concrete}% the concrete-roman, used in concrete mathematics
\usepackage{lipsum,hyperref,datetime2}
\usepackage[dvipsnames]{xcolor}
\hypersetup{
    colorlinks=true,
    filecolor=red,
    linkcolor=blue,
    citecolor=blue,
    urlcolor=blue,
}
% \usepackage{quiver} % for drawing commutative diagrams
\usepackage{graphicx}
% \usepackage{wrapfig}
\graphicspath{ {./graphics/} }
\usepackage[backend=biber,style=alphabetic]{biblatex} % Citations: biber + biblatex
\bibliography{reference.bib} % pdflatex name.tex -> biber name -> pdflatex name.tex
\setcounter{section}{-1}

\newtheoremstyle{mystyle}%                        % Name
  {}%                                             % Space above
  {}%                                             % Space below
  {}%                                             % Body font
  {}%                                             % Indent amount
  {\bfseries}%                                    % Theorem head font
  {}%                                             % Punctuation after theorem head
  { }%                                            % Space after theorem head, ' ', or \newline
  {\thmname{#1}\thmnumber{ #2}\thmnote{ (#3)}}%   % Theorem head spec (can be left empty, meaning `normal')
\numberwithin{equation}{section} % Kirillov thinks this is more usual.
\theoremstyle{mystyle}
\newtheorem{theorem}[equation]{Theorem}
\AtEndEnvironment{theorem}{\null\hfill$\diamond$}%
\newtheorem{convention}[equation]{Convention}
\AtEndEnvironment{convention}{\null\hfill$\diamond$}%
\newtheorem{proposition}[equation]{Proposition}
\AtEndEnvironment{proposition}{\null\hfill$\diamond$}%
\newtheorem{fact}[equation]{Fact}
\AtEndEnvironment{fact}{\null\hfill$\diamond$}%
\newtheorem{definition}[equation]{Definition}
\AtEndEnvironment{definition}{\null\hfill$\diamond$}%
\newtheorem{lemma}[equation]{Lemma}
\AtEndEnvironment{lemma}{\null\hfill$\diamond$}%
\newtheorem{corollary}[equation]{Corollary}
\AtEndEnvironment{corollary}{\null\hfill$\diamond$}%
\newtheorem{example}[equation]{Example}
\AtEndEnvironment{example}{\null\hfill$\diamond$}%
\newtheorem{remark}[equation]{Remark}
\AtEndEnvironment{remark}{\null\hfill$\diamond$}%
\newtheorem{notation}[equation]{Notation}
\AtEndEnvironment{notation}{\null\hfill$\diamond$}%
\let\oldproof\proof
\definecolor{my-dark-gray}{gray}{0.13}
\newcommand{\p}{\textup{\texttt{+}}} % small unary plus
\newcommand{\m}{\mbox{-}} % small unary minus
\renewcommand{\proof}{\color{my-dark-gray}\oldproof}
\renewcommand\qedsymbol{$\blacksquare$}
\newenvironment{Proof}[1][Proof]
  {\proof[#1]\leftskip=0.5cm\rightskip=0cm}
  {\endproof}

\newcommand{\Rep}{\operatorname{Rep}}
\newcommand{\Obj}{\operatorname{Obj}}
\newcommand{\Mor}{\operatorname{Mor}}
\newcommand{\Kar}{\operatorname{Kar}}
\newcommand{\Adm}{\operatorname{Adm}}
\newcommand{\Aut}{\operatorname{Aut}}
\newcommand{\End}{\operatorname{End}}
\newcommand{\Simple}{\operatorname{Simple}}
\newcommand{\id}{\operatorname{id}}
\newcommand{\cok}{\operatorname{cok}}
\newcommand{\im}{\operatorname{im}}

%%%%%%%%%%%%%%%%%% Tikz %%%%%%%%%
\usepackage{tikz}
\usetikzlibrary{shapes.geometric, arrows}
\usetikzlibrary{shapes.multipart} %% for multiple lines in node (must have align=center)
\usetikzlibrary{scopes}
\usetikzlibrary{math}
\usetikzlibrary{decorations.markings,decorations.pathreplacing}
\usetikzlibrary{fadings}
\usepackage{tikz-cd}

\tikzset{
every picture/.style={line width=0.8pt, >=stealth,
                       baseline=-3pt,label distance=-3pt},
%%%%%%%%%%  Node styles
emptynode/.style={circle,minimum size=0pt, inner sep=0pt, outer
sep=0},
dotnode/.style={fill=black,circle,minimum size=2.5pt, inner sep=1pt, outer
sep=0},
small_dotnode/.style={fill=black,circle,minimum size=2pt, inner sep=0pt, outer
sep=0},
morphism/.style={fill=white,circle,draw,thin, inner sep=1pt, minimum size=15pt,
                 scale=0.8},
small_morphism/.style={fill=white,circle,draw,thin,inner sep=1pt,
                       minimum size=10pt, scale=0.8},
ellipse_morphism/.style args={#1}{fill=white,circle,draw,thin,inner sep=1pt,
                       minimum size=5pt, scale=0.8,
												ellipse, draw, rotate=#1},
%note that ellipse stretches based on the text inside, so put \;\;\; in label
coupon/.style={draw,thin, inner sep=1pt, minimum size=18pt,scale=0.8},
semi_morphism/.style args={#1,#2}{
                  fill=white,semicircle,draw,thin, inner sep=1pt, scale=0.8,
                  shape border rotate=#1,
                  label={#1-90:#2}},
%%can only rotate semi_morphisms by right angles tho
%%%% different line styles:
regular/.style={densely dashed}, %% for the regular color, i.e. sum d_i
edge/.style={very thick, draw=green, text=black},
overline/.style={preaction={draw,line width=2mm,white,-}},
thin_overline/.style={preaction={draw,line width=#1 mm,white,-}},
thin_overline/.default=2,
thick_overline/.style={preaction={draw,line width=3mm,white,-}},
really_thick/.style={line width=3mm, gray},
%drinfeld center/.style={>=stealth,green!60!black, double
%distance=1pt,text=black},
boundary/.style={thick,  draw=blue, text=black},
%arrow_decoration={markings, mark=at position 0.5 with {\arrow{>}}}
ribbon/.style={line width=1.5mm, postaction={draw,line width=1mm,white}},
ribbon_u/.style args={#1,#2}{line width=#1mm, postaction={draw,line width=#2mm,white}},
%use line width=0.4pt for thin lines to point to things
%%%%%%% Fill styles %%%%%%%%%%%%%%%
cell/.style={fill=black!10},
subgraph/.style={fill=black!30},
%%%%%%% Mid-path arrows
midarrow/.style={postaction={decorate},
                 decoration={
                    markings,% switch on markings
                    mark=at position #1 with {\arrow{>}},
                 }},
midarrow/.default=0.5,
%%%%% Mid-path arrow but reverse
midarrow_rev/.style={postaction={decorate},
                 decoration={
                    markings,% switch on markings
                    mark=at position #1 with {\arrow{<}},
                 }},
midarrow_rev/.default=0.5,
%%%%% for the flowchart; need align=center to allow multiline in node
block/.style={rectangle, rounded corners, text centered, draw=black, align=center}
}

%% style for flow chart blocks
\tikzstyle{block} = [rectangle, rounded corners, text centered, draw=black, align=center]

%% for shading with gradients
\tikzfading[name=fade inside, inner color=transparent!80, outer
color=transparent!10]
