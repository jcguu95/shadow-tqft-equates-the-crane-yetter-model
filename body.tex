\section{Introduction}

\noindent Among all other dimensions, $4$D is probably the most wild.

\noindent To tackle, tools have been invented and examined.

\noindent Of particular importance and interest are QFTs.

\noindent + Importance of QFTs

\noindent +++ Quantum invariants for knots and manifolds.

\noindent +++ Deep interconnection among representation theory, statistical
physics, and combinatorics.

\noindent +++ Topological phase, quantum computing.. etc.

\noindent +++ Grand unifying scheme.

\noindent + Usefulness and effectiveness (Donaldson and Seiberg-Witten)

\noindent Limitations of current tools and possible strategies (shadow tqft
remains mysterious.)

\noindent Sadly, we showed that they are the same.

\noindent Merit of this work

\noindent + Eliminate doubts
\noindent + Aims to be a thinner introduction to (a part of)
Turaev's book.
\noindent + Survey on $4$D tqft and state sums
\noindent + It's known from Reutter's work that a semisimple
category won't give interesting invariant. More general algebras
are needed, and perhaps in some cases shadows could be easier to
work with.

\noindent Sections Summary

\noindent Some conventions we use (globally) in the paper:
[TODO: make an environment for Conventions?]
\begin{itemize}
  \item We fix a algebraic closed field $\mathbb{k}$ of
        characteristic $0$.
  \item By a vector space $V$ we mean a finite dimensional vector
        space over the field $\mathbb{k}$, unless further
        specified. The linear dual
        $Hom_{\mathbb{k}}(V,\mathbb{k})$ is denoted by
        $V^{\star}$.
\end{itemize}

[TODO: Notes to expert: Use a paragraph to explain what the news
things in this work are.]

% TODO Make sure I refer to the fix version of Turaev's book
% (ISBN and more indicators).

\section{Algebra $(A)$}
\subsection{Premodular category}

The definition of a premodular category is tedious from scratch.
Unfamiliar readers can think of a premodular category roughly as
a higher version of the group algebra of a finite group. The
formal definition (\ref{def/premodular-category}) can be found
after some motivations.

Algebraic objects help abstract tedious details in various
mathematical problems. However, they sometime abstract too much
to recover essential information. In recent years, mathematicians
``categorify'' algebraic objects for them to retain more
information. For example, a (classical) ring is categorified to a
tensor category. [TODO: Insert a reason why tensor categories are
important.]

As in the classical ring theory in which rings with special
properties were of interest, special tensor categories are too of
special interest. For example, a fusion category is a tensor
category that satisfies certain finiteness condition. [TODO:
Insert a reason why fusion categories are important.] While a
fusion category keeps tracks of time-evolution of particles on
$\mathbb{R}^{1}$, a braided fusion category does that for
$\mathbb{R}^{2}$, and is thus a candidate to provide quantum
invariants of knots. A premodular category is a braided fusion
category satisfying a coherence a condition (spherical
structure). Examples include $Rep(G)$ (with $G$ a finite group),
$Rep(\chi)$ (with $\chi$ a finite $2$-group), and
$Rep(U_{q}\mathfrak{g})$ (with $\mathfrak{g}$ a semisimple Lie
algebra, $q$ a root of unity, and $Rep$ the semisimplification of
the category of [..?] representations).

\begin{definition}[premodular category]\label{def/premodular-category}
  A premodular category is a spherical braided fusion category.
\end{definition}

\noindent In particular, a premodular category $C$ is semisimple,
$\mathbb{k}$-linear, and fusion. Define its set of simples
(usually denoted by $I = I_{C}$) to be the set of simple
$C$-objects (up to isomorphism). As taking dual preserves
simplicity, for each $i \in I$ there is a unique element
$i^{\star}$ in $I$ such that $V_{i}^{\star} \in i^{\star}$
whenever $V_{i} \in i$. Using the spherical structure, we can
define for each $i \in I$ the number $dim(i) \in \mathbb{k}$ as
the trace of $id_{V_{i}}$ and the number
$\nu_{i} \in \mathbb{k}^{\star}$ as the twisting coefficient
(which?) (p.76). We further define
$$\Delta_{C} = \sum_{i \in I} \nu_{i}^{-1}dim(i)^{2}.$$

\noindent In order to do computations with a premodular category
we need to choose and fix some extra data. Ultimately, all
intrinsic results are independent of the choice.

\begin{definition}[coordinated premodular
  category]\label{def/coordinated-premodular-category}
  Let $C$ be a premodular category and $I$ its (finite) set of
  simples. Choose a number $D \in \mathbb{k}$ such that
  $D^{2} = \sum_{i \in I} dim(i)^{2}$. Choose a set of
  $C$-objects $\{V_{i}\}_{i \in I}$ such that $V_{i} \in i$ and
  that $V_{0} = \mathbb{1}$, the $\otimes$-identity object.
  Choose a set of isomorphisms [(p.313)]
  $$\{\omega_{i}: V_{i} \to (V_{i^{\star}})^{\star}\}_{i \in I}.$$
  Choose a set of numbers $\{dim'(i) \in \mathbb{k}\}_{i \in I}$
  such that $dim'(0) = 1$, $dim'(i)^{2} = dim(i)$, and
  $dim'(i^{\star}) = dim'(i)$. Choose a set of numbers
  $\{\nu_{i}' \in \mathbb{k}\}_{i \in I}$ such that
  $\nu_{0}' = 1$, $(\nu_{i}')^{2} = \nu_{i}$, and
  $\nu_{i^{\star}}' = \nu_{i}'$ [(p.313)].

  Finally, such a chosen $5$-tuple
  $d = (D, \{V_{i}\}, \{\omega_{i}\}, \{dim'(i)\}, \{\nu'_{i}\})$
  is called a coordinate of the premodular category $C$. Such a
  pair $(C,d)$ is called a coordinated premodular category.
\end{definition}

\noindent We stress again that the data $C$ and $(C,d)$ are not
equivalent. However, we will often confuse the later as a
premodular category because all intrinsic results are independent
of the choice of $d$.

\begin{definition}[multiplicity module]\label{def/multiplicity-module}
  Let $C$ be a premodular category and $I$ its set of simples.
  Respectively, define $H^{ijk}$, $H_{k}^{ij}$, and $H_{ij}^{k}$
  to be the $\mathbb{k}$-modules
  $Hom_{C}(\mathbb{1}, V_{i} \otimes V_{j} \otimes V_{k})$,
  $Hom_{C}(V_{k}, V_{i} \otimes V_{j})$, and
  $Hom_{C}(V_{i} \otimes V_{j}, V_{k})$.
\end{definition}

% \noindent [TODO: need to deal with strictness? Can we really
% use
% coherence theorem to hand-waive this issue?] \\

\noindent Recall that the natural pairing
$\left( H^{ij}_{k} \otimes_{\mathbb{k}} H_{ij}^{k} \to Hom_{C}(V_{k},V_{k}) \xrightarrow{tr} \mathbb{k} \right)$
is nondegenerate by the semisimplicity of $C$. The braided
structure of $C$ guarantees that the $\mathbb{k}$-modules
$H^{ijk}$, $H^{ikj}$, $H^{jik}$, $H^{jki}$, $H^{kij}$, $H^{kji}$
are all isomorphic. In category theory, we must carefully
distinguish equalities from isomorphicities, hence we introduce a
way to keep track of the isomorphisms among the $H^{ijk}$'s.

\begin{definition}[canonical isomorphisms]\label{def/canonical-isomorphism}
  Let $C$ be a premodular category, $c$ its braided structure,
  $I$ its set of simples, and $(i,j,k) \in I^{3}$. Define the
  canonical isomorphisms
  $H^{ijk} \xrightarrow{\sigma_{1}(ijk)} H^{jik}$ and
  $H^{ijk} \xrightarrow{\sigma_{2}(ijk)} H^{ikj}$ by
  $$\sigma_{1}(ijk): \phi \mapsto \nu_{i}'\nu_{j}'(\nu_{k}')^{-1}(c_{V_{i}, V_{j}} \otimes id_{V_{k}})\phi,$$
  $$\sigma_{2}(ijk): \phi \mapsto \nu_{j}'\nu_{k}'(\nu_{i}')^{-1}(id_{V_{i}} \otimes c_{V_{j}, V_{k}})\phi.$$
\end{definition}

\noindent It is a simple exercise in the theory of tensor
categories to check that
\begin{equation} \label{eq1}
  \begin{split}
    \sigma_{1}(jik)\sigma_{1}(ijk) & = id, \\
    \sigma_{2}(ikj)\sigma_{2}(ijk) & = id, \\
    \sigma_{1}(jki)\sigma_{2}(jik)\sigma_{1}(ijk) & = \sigma_{2}(kij)\sigma_{1}(ikj)\sigma_{2}(ijk)
  \end{split}
\end{equation}
so $\sigma_{1}$ and $\sigma_{2}$ specify the isomorphisms among
the six $\mathbb{k}$-modules.

\begin{definition}[symmetrized multiplicity module]\label{def/symmetrized-multiplicity-module}
  Let $C$ be a premodular category, $I$ its set of simples, and
  $i, j, k \in I$. Define the symmetrized multiplicity module
  $H(i,j,k)$ to be the $\mathbb{k}$-module consisting of
  functions $\phi$ that assign an element
  $\phi^{i_{1}i_{2}i_{3}} \in H^{i_{1}i_{2}i_{3}}$ to each
  ordering $(i_{1}, i_{2}, i_{3})$ of the set $\{i, j, k\}$.
\end{definition}

\noindent We stress that all the symmetrized modules
$H(i,j,k), H(i,k,j), H(j,i,k)$, $H(j,k,i), H(k,i,j), H(k,j,i)$
are all \textit{equal as sets}.

\begin{definition}[contraction]\label{def/contraction}
  Let $C$ be a (coordinated) premodular category, $I$ its set of
  simples, and $i, j, k \in I$. Define the contraction map
  $H^{ijk} \otimes H^{k^{\star}j^{\star}i^{\star}} \to \mathbb{k}$
  by the following diagram [TODO: insert diagram as in qiok
  figure VI.3.5]. Denote the canonically induced contraction map
  on the symmetrized modules to be (p.334)
  $$\ast_{ijk}: H(i,j,k) \otimes_{\mathbb{k}} H(i^{\star}, j^{\star}, k^{\star}) \to \mathbb{k}.$$
  This defines a nondegenerate pairing on the domain of
  $\ast_{ijk}$ and thus also induces a canonical element
  $Id(i,j,k)$ there (p.333).
\end{definition}

\noindent When there is little danger of confusion, we will abuse
notation by denoting natural contractions from the non-ordered
tensor products
$V \otimes_{\mathbb{k}} H(i,j,k) \otimes_{\mathbb{k}} H(i^{\star}, j^{\star}, k^{\star})$
to $\mathbb{k}$ by $\ast_{ijk}$ where $V$ is any
$\mathbb{k}$-module.

\subsection{$6$j-symbol, $10$j-symbol, and $15$j-symbol}

\newcommand{\sixJSymbol}[6]{\begin{vmatrix}
  #1 & #2 & #3 \\
  #4 & #5 & #6 \\
\end{vmatrix}}

\begin{definition}[$6$j-symbol]\label{def/6j-symbol}
  For each $(i,j,k,l,m,n) \in I^{6}$, we define the (normalized)
  $6$j symbol
  $$\sixJSymbol{i}{j}{k}{l}{m}{n}$$
  to be the invariant of framed graph (cf \cite[section
  VI.4]{turaev-qiok-3-manifolds})
  $\mathbb{F}(\Gamma) \in Hom_{\mathbb{k}}(M, \mathbb{k})$, where
  $\Gamma$ denotes the graph depicts in [TODO: graphic? Add
  graphic.], and $M$ denotes the non-ordered tensor product over
  $\mathbb{k}$ of the $\mathbb{k}$-modules
  $H(i^{\star}, j^{\star}, k)$, $H(i, m^{\star}, n)$,
  $H(j, l, n^{\star})$, $H(k^{\star}, l^{\star}, m)$.
\end{definition}

We list some basic equalities among $6$j symbols.

\begin{proposition}[basic equalities of $6$j symbols]\label{prop/basic-equalities-of-6j-symbols}
  Let $C$ be a (coordinated) premodular category, $I$ its set of
  simples, $i, j, k, k', l, m \in I$,
  $j_{0}, j_{1}, \ldots, j_{8} \in I$, and $\delta$ the Kronecker
  delta. Then we have the degenerated $6$j symbol
  \begin{equation}\label{eqn/degenerated-6j}
    \sixJSymbol{i}{j}{k}{l}{m}{0} = \delta_{m,i} \delta_{l,j^{\star}}dim'(i)^{-1}dim'(j)^{-1}Id(i,j,k^{\star}).
  \end{equation}
  We also have the so called Biedenharn-Elliott identity as an equality in the non-ordered tensor product of the $\mathbb{k}$-modules
  $$
  H(j_{3}^{\star}, j_{5}^{\star}, j_{6}),\,
  H(j_{1}^{\star}, j_{2}^{\star}, j_{5}),\,
  H(j_{4}^{\star}, j_{6}^{\star}, j_{0}),\,
  H(j_{0}^{\star}, j_{1}, j_{7}),\,
  H(j_{7}^{\star}, j_{2}, j_{8}),\,
  H(j_{8}^{\star}, j_{3}, j_{4})
  $$
  (in the context of state sum over a triangulation, this corresponds to the $(2,3)$ Pachner move):
  \begin{equation}\label{eqn/Biedenharn-Elliot-identity}
    \ast_{j_{0}^{\star}j_{5}j_{8}}
    \left(
      \sixJSymbol{j_{5}}{j_{3}}{j_{6}}{j_{4}}{j_{0}}{j_{8}} \otimes \sixJSymbol{j_{1}}{j_{2}}{j_{5}}{j_{8}}{j_{0}}{j_{7}}
    \right)
    =
    \sum_{j \in I} dim(j)
    \ast_{j^{\star}j_{2}j_{3}}\ast_{jj_{4}j_{7}^{\star}}\ast_{jj_{1}j_{6}^{\star}}
    \left(
      \sixJSymbol{j_{1}}{j_{2}}{j_{5}}{j_{3}}{j_{6}}{j} \otimes
      \sixJSymbol{j_{1}}{j}{j_{6}}{j_{4}}{j_{0}}{j_{7}} \otimes
      \sixJSymbol{j_{2}}{j_{3}}{j}{j_{4}}{j_{7}}{j_{8}}
    \right).
  \end{equation}
  We also have the orthonormality relation
  \begin{equation}\label{eqn/orthonormality-relation}
    \delta_{k,k'} Id(i,j,k^{\star}) \otimes Id(k,l,m^{\star})
    =
    dim(k)\,\sum_{n \in I} dim(n) \ast_{im^{\star}n} \ast_{jln^{\star}}
    \left(
      \sixJSymbol{i^{\star}}{j^{\star}}{k^{\star}}{l^{\star}}{m^{\star}}{n^{\star}} \otimes
      \sixJSymbol{i}{j}{k'}{l}{m}{n}
    \right).
  \end{equation}
  Finally, we have the Racah identity
  \begin{equation}\label{eqn/Racah-identity}
    \nu'_{j_{3}} \nu'_{j_{6}} (\nu'_{j_{1}} \nu'_{j_{2}} \nu'_{j_{4}} \nu'_{j_{5}})^{-1}
    \sixJSymbol{j_{1}}{j_{2}}{j_{3}}{j_{4}}{j_{5}}{j_{6}}
    =
    \sum_{j \in I} (\nu'_{j})^{-1} dim(j) \ast_{j^{\star}j_{1}j_{4}} \ast_{jj_{2}j_{5}^{\star}}
    \left(
      \sixJSymbol{j_{1}}{j_{4}}{j}{j_{2}}{j_{5}}{j_{6}} \otimes
      \sixJSymbol{j_{2}}{j_{1}}{j_{3}}{j_{4}}{j_{5}}{j}
    \right)
  \end{equation}
\end{proposition}

\begin{proof}
  Proof?
  [TODO: refer to Turaev but stress that they also work for premodular categories.] (p.334)
\end{proof}

\begin{definition}[$10$j-symbol]\label{def/10j-symbol}
  [TODO]
\end{definition}

\begin{definition}[$15$j-symbol]\label{def/15j-symbol}
  [TODO]
\end{definition}

% \subsection{Example}
% tables in appendix?

\section{Topology $(T)$}
\subsection{$4$-manifold}

[TODO: Draw a diagram that relating different kinds of data.]

Manifolds in real dimension four are of interest because their
wildness, witnessed in the following examples:

\begin{enumerate}
  \item Real dimension $4$ is the first dimension in which the
        topological structure, the piecewise-linear structure,
        and the smooth structure cease to coincide.
  \item The euclidean space $\mathbb{R}^{n}$ has exactly one
        smooth structure (up to diffeomorphism) for each
        non-negative integer $n$ except for $(n=4)$, in which
        case (strictly) more than a countably infinite worth of
        inequivalent structures exist \cite{?}.
  \item The Poincare conjecture for the $n$-dimensional sphere
        $S^{n}$ has been resolved except for $n=4$, which remains
        widely open to date \cite{?}.
  \item The physical universe in which we live seems to be
        well-modeled by a $4$-manifold.
\end{enumerate}

\noindent Despite the wildness, Kirby and Siebenmann
\cite{kirby-siebenmann} \cite{turaev-qiok-3-manifolds} show that
the category of smooth manifolds is equivalent to the category of
piecewise-linear manifolds. The data of the later are much
smaller and more elegant. Thus really we will be working on the
piecewise-linear cases, and hence by a manifold we mean a
piecewise-linear, oriented and closed manifold in real dimension
$4$ unless further specified. Readers who are uncomfortable with
the result of Kirby-Siebenmann can treat this entire work as one
about piecewise-linear manifolds instead of the smooth manifolds.

\subsection{Triangulation} \label{subsection/triangulation}

\noindent The standard oriented $4$-simplex is defined to be
following topological subspace (with natural orientation)
$$\Delta = \Delta_{4} = \left\{x \in [0,1]^{5} \,|\, \sum x_{i} = 1 \right\}.$$
By a triangulated ($4$-)space we mean a finite collection of
copies of $\Delta$ with a finite collection of (affine and
oriented) identifications of the pairs of distinct faces. Clearly
an oriented topological space canonically arises from a
triangulated space. We shall confuse both of them. By a
triangulation of a $4$-manifold $X$ we mean a pair of
triangulated space $X'$ and a piecewise-linear homeomorphism
$X \simeq X'$. Two triangulated $4$-spaces give rise to the same
manifold (up to diffeomorphism) if and only if both gluing data
can be relate by a finite sequence of Pachner basic moves
\cite{?}. Thus we can present a smooth $4$-manifold (up to
diffeomorphism) by a gluing datum (up to Pachner moves). As we
will see, the Crane-Yetter state sum transforms the gluing datum
into a number.

\subsection{Handle decomposition}

\noindent By Morse's theory of extremal points, any smooth
manifold admits a handle decomposition. By Cerf theory, two
handle decompositions present the same manifold (up to
diffeomorphism) if and only if both decomposition data are
related by a finite sequence of handle creations, handle
annihilations, and handle slides \cite{?}. A triangulation of a
manifold admits a natural handle decomposition by taking dual.
The correct state sum is the universal state sum \cite{?}; it
transforms the decomposition datum into a number.

[TODO: Mention the reconstruction theorem for closed
4-manifolds.]

\subsection{Shadow}

% TODO - Figure out why we need integer-shadows.

\noindent A shadow is another type of structure that encodes
closed $4$-manifolds. We will briefly recall it, and describe how
it presents closed $4$-manifolds in
\ref{def/shadow-of-a-4-manifold}.

\noindent Roughly speaking, a shadow is a $2$-polyhedron with
extra decorations (called gleams) that remember the twisting
data. A $2$-polyhedron is a topological and combinatorial object
that encodes $3$-dimensional manifolds \cite{?Matveev}. It is
called a pre-foam in the modern literature
\cite{khovanov-robert/foam}.

\begin{definition}[tripod]\label{def/tripod}
  Define the standard tripod to be the topological subspace of
  $\mathbb{R}^{3}$ consisting of the points $(x,y,z)$ such that
  at least two of the entries are zero, and the last entry
  belongs to $[0,1)$. Define a tripod to be any topological space
  homeomorphic to the standard tripod.
\end{definition}

\begin{definition}[cone]\label{def/cone}
  For each topological space $X$, define its standard open cone
  $cone(X)$ to be the quotient space
  $(X \times \mathbb{R_{\geq 0}})/((x,0) \sim (x',0)).$ Define an
  open cone of $X$ to be any topological space homeomorphic to
  $cone(X)$.
\end{definition}

\begin{definition}[local shape]\label{def/local-shape}
  Let $X$ be a topological space and $x \in X$. Denote by $T$ the
  standard tripod and $S$ the $1$-skeleton of the boundary of the
  standard tetrahedron (a trivalent graph with $4$ vertices and
  $6$ edges). Respectively, we say that $x$ is a smooth point, a
  line point, a tetrahedral point, a boundary smooth point, or a
  boundary line point of $X$ if it has a relative neighborhood
  homeomorphic to $(\mathbb{R}^{2},0)$,
  $(T \times \mathbb{R}, (0, 0))$, $(cone(S), (*, 0))$,
  $(\mathbb{R} \times \mathbb{R}_{\geq 0}, (0, 0))$, or
  $(T \times \mathbb{R}_{\geq 0}, (0, 0))$.
\end{definition}

\begin{definition}[simple $2$-polyhedron]\label{def/simple-2-polyhedron}
  A simple $2$-polyhedron with boundary is defined to be a
  piecewise-linear compact CW-complex $P$ of real dimension two,
  such that each of its point $p$ is either a smooth point, a
  line point, a tetrahedral point, a boundary smooth point, or a
  boundary line point. If only the first three types are
  involved, we call $P$ a simple $2$-polyhedron without boundary.
\end{definition}

\begin{definition}[components of a simple $2$-polyhedron]\label{def/components-of-a-simple-2-polyhedron}
  Let $P$ be a simple $2$-polyhedron with boundary. Define the
  set of smooth points (or called interior points) of $P$ to be
  $Int(P)$. Define the set of line points, tetrahedral points,
  and boundary line points to be $sing(P)$. Define the set of
  boundary line points and boundary smooth points to be
  $\partial P$. Call a connected component of $Int(P)$ to be a
  region of $P$; define the set of regions to be $Region(P)$. $P$
  is said to be orientable if each region of $P$ is orientable.
  An orientation of $P$ is an assignment of orientations to each
  of the region.
\end{definition}

\begin{definition}[shadowed $2$-polyhedron]\label{def/shadowed-2-polyhedron}
  Let $P$ be a simple $2$-polyhedron, and $A$ an abelian group
  with a distinguished element $\omega \in A$. We define a shadow
  to be a pair of an orientable $2$-polyhedron $P$ and a map
  (called gleam) $gl: Region(P) \to A$. Unless specified further,
  we assume that $A = \mathbb{Z}\left[\frac{1}{2}\right]$ and
  $\omega = \frac{1}{2}$.

  \noindent We denote $-P$ to be the same simple $2$-polyhedron
  but with all gleams flipped by $(a \mapsto -a)$.
\end{definition}

For each connected oriented closed surface $\Sigma$ and each
$a \in A$, there is a shadowed $2$-polyhedron $\Sigma_{a}$ which
consists of $\Sigma$ with the gleam $a$ assigned to the only
region. For example, the $0$-gleamed $2$-sphere is denoted as
$S^{2}_{0}$.

\begin{definition}[shadow moves]\label{def/shadow-moves}
  [TODO]
Basic moves
  (P1)
  (P2)
  (P3)
  Turaev p369

  A general is a finite composition of $P_1^{\pm 1}$..
\end{definition}

\begin{definition}[shadow]\label{def/shadow}
  A shadow is an equivalence class of shadowed $2$-polyhedron $P$
  up to a shadow move. We denote the shadow by $[P]$, and say
  that $P$ represents the shadow $[P]$. (p.370)
\end{definition}

\noindent For two connected shadow $[P]$ and $[P']$, we construct
the shadow $[P]+[P']$ as follows. Arbitrarily identify two
arbitrarily chosen closed disks $D \subset Int(P)$ and
$D' \subset Int(P')$ in $P \coprod P'$, and equip the interior of
$D$ (a new region) with gleam $0$. So defines a simple
$2$-polyhedron and we say that it represents $[P]+[P']$. It is
well-defined by \cite[lemma VIII.2.1.1]{turaev-}. For an integer
$m \in \mathbb{Z_{\geq 0}}$, we define $m[P]$ as the sum of
$m$-many $[P]$.

\begin{definition}[stable shadow]\label{def/stable-shadow}
  Two connected shadowed polyhedra $P$, $P'$ are called stably
  shadow equivalent if there exists
  $n, n' \in \mathbb{Z_{\geq 0}}$ such that
  $[P] + m[S^{2}_{0}] = [P'] + m'[S^{2}_{0}]$. Extend the
  definition to non-connected ones in an obvious fashion. A
  stable shadow is defined to be a shadowed polyhedron up to
  stable shadow equivalence. Denote the stable shadow of $[P]$ to
  be $stab([P])$.
\end{definition}

\noindent We are ready to present a closed $4$-manifold in terms
of shadows.

\begin{definition}[locally flat $2$-polyhedron in a
  $4$-manifold]\label{def/locally-flat-2-polyhedron-in-a-4-manifold}
  Let $X$ be a closed $4$-manifold. A $2$-polyhedron $P$ in $X$
  is flat at a point $p \in P$ if there exists a neighborhood $U$
  of $p$ in $X$ such that $U \cap P$ lies in a $3$-dimensional
  submanifold of $X$. We say that $P$ is locally flat if it is
  flat at all $p \in P$. (p.394)
\end{definition}

\begin{definition}[skeleton of a $4$-manifold]\label{def/skeleton-of-a-4-manifold}
  Let $X$ be a closed $4$-manifold. A skeleton (p.395) of $X$ is
  a locally flat orientable simple $2$-polyhedron without
  boundary $P$ such that a closed regular neighborhood of it with
  some $3$- and $4$-handles form $X$.
\end{definition}

\noindent For example, the $\mathbb{C}P^{1}$ (standardly
embedded) is a skeleton of $\mathbb{C}P^{2}$. By [turaev theorem
IX.1.5.], every $4$-manifold has a skeleton (by compressing the
$(0,1,2)$-handles in an arbitrary handle decomposition).

\begin{definition}[stable shadow of a $4$-manifold]\label{def/stable-shadow-of-a-4-manifold}
  Let $X$ be a closed $4$-manifold. Take a skeleton $P$ of $X$
  and construct a shadowed simple $2$-polyhedron by assigning
  gleams to the regions $\Sigma$ in the following way.
  \begin{enumerate}
    \item If $\Sigma$ is homeomorphic to a closed surface, define
          the gleam to be the self-intersection (which is
          independent of the orientation of $\Sigma$)
          $$([\Sigma] \cdot [\Sigma]) \in H_{0}(X;\mathbb{Z}) = \mathbb{Z} \subset \mathbb{Z}\left[1/2\right].$$
    \item Otherwise, $\Sigma$ is non-compact. Deformation retract
          it to a compact subsurface $\Sigma_{0}$. Denote $N$ to
          be the normal bundle of $\Sigma_{0}$ in $X$. Consider
          the line bundle $l$ over $\partial \Sigma_{0}$ by
          [TODO, see VIII.6.2. and p.397], which may be regarded
          as a subbundle of $N|_{\partial \Sigma_{0}}$. The
          circle bundle $\mathbb{P}(N)$ is trivial over
          $\Sigma_{0}$ since the later is of $1$-homotopy type.
          With a choice of an orientation of $\Sigma_{0}$ and
          $X$, $l$ induces a section of
          $\mathbb{P}(N)|_{\partial}$. The obstruction class of
          this section to the whole $\mathbb{P}(N)$ is an element
          of
          $H^{2}(\Sigma_{0}, \partial \Sigma_{0}; \pi_{1}(S^{1})) = \mathbb{Z}$.
          % TODO: Understand this part really.
          Finally, define the gleam to be the half of the
          resulting integer (which is independent to the choice
          of $\Sigma_{0}$).
  \end{enumerate}
  It is the main theorem of \cite[chapter IX]{turaev} (see
  IX.1.7) that all shadowed polyhedra chosen in such fashion
  above are all stably shadow equivalent. Therefore, it defines
  the stable shadow $sh(X)$ of the closed $4$-manifold $X$.
\end{definition}

\noindent Examples include
$sh(\pm\mathbb{C}P^{2}) = stab([S^{2}_{\pm 1}])$ and
$sh(S^{4}) = stab([S^{2}_{0}])$.

\noindent Recall from \cite[section IX. 4]{turaev} that a handle
decomposition of a closed $4$-manifold $X$ gives rise to a shadow
of $X$. The explicit construction will be recalled below as it
will be used to prove our main theorem. We need some
preliminaries.

\begin{definition}[skeleton of a $3$-manifold]\label{def/skeleton-of-a-3-manifold}
  Let $Y$ be a closed $3$-manifold. A skeleton of $Y$ is an
  orientable simple $2$-polyhedron without boundary $P \subset Y$
  such that $Y \ P$ is a disjoint union of open $3$-balls.
  (p.400)
\end{definition}

\begin{definition}[shadow cone of a framed link in a
  $3$-manifold]\label{def/shadow-cone-of-a-framed-link-in-a-3-manifold}
  \noindent Every compact $3$-manifold $Y$ has a skeleton (turaev
  IX 2.1.1). For example, the equator ($S^{2}$) of $S^{3}$ is a
  skeleton. Let $P$ be a skeleton of $Y$ and $l$ be a framed link
  in $Y$. Projecting $l$ generically onto $P$ induces a shadow
  projection. Assign gleams around each crossing point as in
  (turaev figure IX.3.4). Then construct the shadow as in
  (IX.3.3) by naturally attaching a disk along each projected
  component on $P$ (as a new region) endowed with zero gleam.
  Denote the resulting shadow to be $CO(Y,l)$. [TODO: up to what
  is it well-defined + reference?]
\end{definition}

\noindent We are ready to construct a shadow of a $4$-manifold
from a handle decomposition.

\begin{definition}[shadow of a $4$-manifold from handle
  decomposition]\label{def/shadow-of-a-4-manifold-from-handle-decomposition}
  Let $X$ be an oriented $4$-manifold and
  $H = \bigcup_{i=0}^{4} H_{i}$ be a handle decomposition, where
  $H_{i}$ denotes the union of the handles of index $i$. Define
  $Y$ to be the closed $3$-manifold $\partial(H_{0} \cup H_{1})$.
  By the definition of handle decomposition, the gluing datum of
  $H_{2}$ onto the lower parts is encoded as a link $l$ in $Y$.
  Define the stable shadow $sh'(X,H)$ to be $CO(Y,l)$.
\end{definition}

\noindent It is a theorem (IX.4.2) that $sh'(X,H)$ does not
depend on the choice of $H$. In fact, $sh'(X,H)$ equals the
stable shadow of $sh(X)$ (IX.7).

\begin{remark}
  It is not surprising why the handles of indices $3$ and $4$ are
  not included in the construction of the shadow. Indeed, the
  handles of indices $\leq 2$ are enough to reconstruct the whole
  closed $4$-manifold (cite: Stern or Fintushel?).
\end{remark}

\section{Sum $\left( \int_{T}{A} \right)$}
\subsection{Crane-Yetter state sum}

\subsection{Shadow state sum}

[..]

\begin{proposition}[shadow state sum is local]\label{prop/shadow-state-sum-is-local}
  [TODO]
\end{proposition}

We end the subsection with a key lemma.

\begin{lemma}[$10$j symbol as a shadow state sum]\label{lemma/10j-symbol-as-shadow-state-sum}
  [TODO]
\end{lemma}

\subsection{Main result: equivalence of state sums}

\noindent We state and prove the main theorem of this paper.

\begin{theorem}[equivalence of state sums]
  Let $X$ be a closed $4$-manifold and $C$ be a (coordinated)
  premodular category. Then $$\int^{CY}_{X}C = \int^{sh}_{X} C.$$
\end{theorem}

\begin{proof}
  Let $1 >> \epsilon > 0$, $T$ be a triangulation of $X$, and
  $H = \bigcup_{i=0}^{4} H_{i}$ be a dual handle decomposition in
  which the attaching region of each $1$-handle on each
  $0$-handle is the $\epsilon$-ball ($3$-dimensional) of the
  central point of corresponding $3$-cell. (This makes sense
  because we have realized the standard $4$-cell as a subspace of
  the standard $\mathbb{R}^{5}$ in
  (\ref{subsection/triangulation}).) To compute the shadow state
  sum, we first need to construct a shadow for $X$. We will
  construct it by the procedure given in
  \ref{def/shadow-of-a-4-manifold-from-handle-decomposition} as
  the resulting shadow will resemble the structure of $T$.

  Suppose $T$ has $n$ $4$-cells. Let $\Delta$ be a $4$-cell in
  $T$. It corresponds to a $0$-handle $h_{0}$. Each of the
  $3$-cell $\delta$ among the five in $\partial \Delta$
  corresponds to (a half of) a $1$-handle $h_{1}$, which is glued
  to $h_{0}$. Hence $H_{0} \cup H_{1}$ is a solid $4$-body with
  $n$ $0$-cells and $\frac{5n}{2}$ $1$-cells. The $3$-manifold
  $Y = \partial(H_{0} \cup H_{1})$ in the
  construction \ref{def/shadow-of-a-4-manifold-from-handle-decomposition}
  is therefore homeomorphic to the space constructed as follows:
  First take the union of $n$ $3$-spheres identified to the
  $0$-cells, and connected sum any pair of them if and only if
  the corresponding $0$-cells are connected by $1$-cells. It is
  also homeomorphic to a union of $n$ $S^{3} - B^{3} \times 5$
  (denoted by $S_{0,5}$) with obvious parts that overlap, where
  the $B^{3}$ can be identified as the $\epsilon$-ball introduced
  above.

  Following the construction, we will construct a skeleton for
  the $3$-manifold $Y$. The construction will be first carried
  out locally in each $S_{0,5}$, and then glued together with
  some modification:

  \begin{enumerate}
    \item In each $S_{0,5}$, fix a $3$-ball $B$ of size
          $\epsilon$ such that the following condition holds. For
          each removed $B^{3}$, pick a shortest path from its
          center to the center of $B$. Thicken each path to width
          $\frac{\epsilon}{2}$. We require that the five thicken
          paths do not intersect each other. The boundary of the
          union is a $5$-punctured sphere $\Sigma_{0,5}$ in
          $S_{0,5}$.
    \item Identify arbitrarily the end of a $5$-punctured sphere
          (a circle) with the end of another $5$-punctured sphere
          (another circle) if this pair corresponds to the same
          $1$-handle.
    \item For each identified circle, glue a $2$-disk along its
          boundary.
  \end{enumerate}

  The resulting polyhedron $P$ is simple and orientable.
  Moreover, its complement in $Y$ is a union of open $3$-balls.
  Therefore, $P$ is a skeleton for the $3$-manifold $Y$ as
  desired.

  The next step in the construction
  (\ref{def/shadow-of-a-4-manifold-from-handle-decomposition}) is
  to project the link $l \subset Y$ (as the gluing datum of
  $H_{2}$ onto $H_{0} \cup H_{1}$). As the handle decomposition
  $H$ comes from a triangulation, the links are identical in each
  local piece $S^{3} - B^{3} \times 5$. A simple geometric
  exercise shows that a generic projection of $l$ onto $P$ looks
  as follows (with gleams added as in the construction
  \ref{def/shadow-of-a-4-manifold-from-handle-decomposition}).

  \noindent [TODO: Add graphic [20220217T143000 (1)]]

  Using \ref{prop/shadow-state-sum-is-local}, it remains to show
  that the shadow state sum of the (local) shadow depicted above
  coincides with the $10$j symbol. This is exactly what the key
  lemma \ref{lemma/10j-symbol-as-shadow-state-sum} shows, so we
  are done.

\end{proof}
