\section{Introduction}

\noindent Among all other dimensions, $4$D is probably the most wild.

\noindent To tackle, tools have been invented and examined.

\noindent Of particular importance and interest are QFTs.

\noindent + Importance of QFTs

\noindent +++ Quantum invariants for knots and manifolds.

\noindent +++ Deep interconnection among representation theory, statistical
physics, and combinatorics.

\noindent +++ Topological phase, quantum computing.. etc.

\noindent +++ Grand unifying scheme.

\noindent + Usefulness and effectiveness (Donaldson and Seiberg-Witten)

\noindent Limitations of current tools and possible strategies (shadow tqft
remains mysterious.)

\noindent Sadly, we showed that they are the same.

\noindent Merit of this work

\noindent + Eliminate doubts
\noindent + Perhaps shadows are easier to work with in certain cases.
\noindent + Survey on $4$D tqft and state sums

\noindent Sections Summary

\noindent Some conventions we use (globally) in the paper:
\begin{itemize}
  \item We fix a algebraic closed field $\mathbb{k}$ of
        characteristic $0$.
  \item By a vector space $V$ we mean a finite dimensional vector
        space over the field $\mathbb{k}$, unless further
        specified. The linear dual
        $Hom_{\mathbb{k}}(V,\mathbb{k})$ is denoted by
        $V^{\star}$.
\end{itemize}

\section{Algebra $(A)$}
\subsection{Premodular category}

The definition of a premodular category is tedious from scratch.
Unfamiliar readers can think of a premodular category roughly as
a higher version of the group algebra of a finite group. The
formal definition (\ref{def/premodular-category}) can be found
after some motivations.

Algebraic objects help abstract tedious details in various
mathematical problems. However, they sometime abstract too much
to recover essential information. In recent years, mathematicians
``categorify'' algebraic objects for them to retain more
information. For example, a (classical) ring is categorified to a
tensor category. [TODO: Insert a reason why tensor categories are
important.]

As in the classical ring theory in which rings with special
properties were of interest, special tensor categories are too of
special interest. For example, a fusion category is a tensor
category that satisfies certain finiteness condition. [TODO:
Insert a reason why fusion categories are important.] While a
fusion category keeps tracks of time-evolution of particles on
$\mathbb{R}^{1}$, a braided fusion category does that for
$\mathbb{R}^{2}$, and is thus a candidate to provide quantum
invariants of knots. A premodular category is a braided fusion
category satisfying a coherence a condition (spherical
structure). Examples include $Rep(G)$ (with $G$ a finite group),
$Rep(\chi)$ (with $\chi$ a finite $2$-group), and
$Rep(U_{q}\mathfrak{g})$ (with $\mathfrak{g}$ a semisimple Lie
algebra, $q$ a root of unity, and $Rep$ the semisimplification of
the category of [..?] representations).

\begin{definition}[premodular category]\label{def/premodular-category}
  A premodular category is a spherical braided fusion category.
\end{definition}

\noindent In particular, a premodular category $C$ is semisimple,
$\mathbb{k}$-linear, and fusion. Define its set of simples
(usually denoted by $I = I_{C}$) to be the set of simple
$C$-objects (up to isomorphism). As taking dual preserves
simplicity, for each $i \in I$ there is a unique element
$i^{\star}$ in $I$ such that $V_{i}^{\star} \in i^{\star}$
whenever $V_{i} \in i$. Using the spherical structure, we can
define for each $i \in I$ the number $dim(i) \in \mathbb{k}$ as
the trace of $id_{V_{i}}$ and the number
$\nu_{i} \in \mathbb{k}^{\star}$ as the twisting coefficient
(which?) (p.76). We further define
$$\Delta_{C} = \sum_{i \in I} \nu_{i}^{-1}dim(i)^{2}.$$

\noindent In order to do computations with a premodular category
we need to choose and fix some extra data. Ultimately, all
intrinsic results are independent of the choice.

\begin{definition}[coordinated premodular
  category]\label{def/coordinated-premodular-category}
  Let $C$ be a premodular category and $I$ its (finite) set of
  simples. Choose a number $D \in \mathbb{k}$ such that
  $D^{2} = \sum_{i \in I} dim(i)^{2}$. Choose a set of
  $C$-objects $\{V_{i}\}_{i \in I}$ such that $V_{i} \in i$ and
  that $V_{0} = \mathbb{1}$, the $\otimes$-identity object.
  Choose a set of isomorphisms [(p.313)]
  $$\{\omega_{i}: V_{i} \to (V_{i^{\star}})^{\star}\}_{i \in I}.$$
  Choose a set of numbers $\{dim'(i) \in \mathbb{k}\}_{i \in I}$
  such that $dim'(0) = 1$, $dim'(i)^{2} = dim(i)$, and
  $dim'(i^{\star}) = dim'(i)$. Choose a set of numbers
  $\{\nu_{i}' \in \mathbb{k}\}_{i \in I}$ such that
  $\nu_{0}' = 1$, $(\nu_{i}')^{2} = \nu_{i}$, and
  $\nu_{i^{\star}}' = \nu_{i}'$ [(p.313)].

  Finally, such a chosen $5$-tuple
  $d = (D, \{V_{i}\}, \{\omega_{i}\}, \{dim'(i)\}, \{\nu'_{i}\})$
  is called a coordinate of the premodular category $C$. Such a
  pair $(C,d)$ is called a coordinated premodular category.
\end{definition}

\noindent We stress again that the data $C$ and $(C,d)$ are not
equivalent. However, we will often confuse the later as a
premodular category because all intrinsic results are independent
of the choice of $d$.

\begin{definition}[multiplicity module]\label{def/multiplicity-module}
  Let $C$ be a premodular category and $I$ its set of simples.
  Respectively, define $H^{ijk}$, $H_{k}^{ij}$, and $H_{ij}^{k}$
  to be the $\mathbb{k}$-modules
  $Hom_{C}(\mathbb{1}, V_{i} \otimes V_{j} \otimes V_{k})$,
  $Hom_{C}(V_{k}, V_{i} \otimes V_{j})$, and
  $Hom_{C}(V_{i} \otimes V_{j}, V_{k})$.
\end{definition}

% \noindent [TODO: need to deal with strictness? Can we really
% use
% coherence theorem to hand-waive this issue?] \\

\noindent Recall that the natural pairing
$\left( H^{ij}_{k} \otimes_{\mathbb{k}} H_{ij}^{k} \to Hom_{C}(V_{k},V_{k}) \xrightarrow{tr} \mathbb{k} \right)$
is nondegenerate by the semisimplicity of $C$. The braided
structure of $C$ guarantees that the $\mathbb{k}$-modules
$H^{ijk}$, $H^{ikj}$, $H^{jik}$, $H^{jki}$, $H^{kij}$, $H^{kji}$
are all isomorphic. In category theory, we must carefully
distinguish equalities from isomorphicities, hence we introduce a
way to keep track of the isomorphisms among the $H^{ijk}$'s.

\begin{definition}[canonical isomorphisms]\label{def/canonical-isomorphism}
  Let $C$ be a premodular category, $c$ its braided structure,
  $I$ its set of simples, and $(i,j,k) \in I^{3}$. Define the
  canonical isomorphisms
  $H^{ijk} \xrightarrow{\sigma_{1}(ijk)} H^{jik}$ and
  $H^{ijk} \xrightarrow{\sigma_{2}(ijk)} H^{ikj}$ by
  $$\sigma_{1}(ijk): \phi \mapsto \nu_{i}'\nu_{j}'(\nu_{k}')^{-1}(c_{V_{i}, V_{j}} \otimes id_{V_{k}})\phi,$$
  $$\sigma_{2}(ijk): \phi \mapsto \nu_{j}'\nu_{k}'(\nu_{i}')^{-1}(id_{V_{i}} \otimes c_{V_{j}, V_{k}})\phi.$$
\end{definition}

\noindent It is a simple exercise in the theory of tensor
categories to check that
\begin{equation} \label{eq1}
  \begin{split}
    \sigma_{1}(jik)\sigma_{1}(ijk) & = id, \\
    \sigma_{2}(ikj)\sigma_{2}(ijk) & = id, \\
    \sigma_{1}(jki)\sigma_{2}(jik)\sigma_{1}(ijk) & = \sigma_{2}(kij)\sigma_{1}(ikj)\sigma_{2}(ijk)
  \end{split}
\end{equation}
so $\sigma_{1}$ and $\sigma_{2}$ specify the isomorphisms among
the six $\mathbb{k}$-modules.

\begin{definition}[symmetrized multiplicity module]\label{def/symmetrized-multiplicity-module}
  Let $C$ be a premodular category, $I$ its set of simples, and
  $i, j, k \in I$. Define the symmetrized multiplicity module
  $H(i,j,k)$ to be the $\mathbb{k}$-module consisting of
  functions $\phi$ that assign an element
  $\phi^{i_{1}i_{2}i_{3}} \in H^{i_{1}i_{2}i_{3}}$ to each
  ordering $(i_{1}, i_{2}, i_{3})$ of the set $\{i, j, k\}$.
\end{definition}

\noindent We stress that all the symmetrized modules
$H(i,j,k), H(i,k,j), H(j,i,k)$, $H(j,k,i), H(k,i,j), H(k,j,i)$
are all \textit{equal as sets}.

\subsection{$6$j-symbol, $10$j-symbol, and $15$j-symbol}

\begin{definition}[$6$j-symbol]\label{def/6j-symbol}
  For each $(i,j,k,l,m,n) \in I^{6}$, we define the (normalized)
  $6$j symbol
  $$
  \begin{vmatrix}
    i & j & k  \\
    l & m & n  \\
  \end{vmatrix}
  $$
  to be the invariant of framed graph (cf \cite[section
  VI.4]{turaev-qiok-3-manifolds})
  $\mathbb{F}(\Gamma) \in Hom_{\mathbb{k}}(M, \mathbb{k})$, where
  $\Gamma$ denotes the graph depicts in [TODO: graphic? Add
  graphic.], and $M$ denotes the non-ordered tensor product over
  $\mathbb{k}$ of the $\mathbb{k}$-modules
  $H(i^{\star}, j^{\star}, k)$, $H(i, m^{\star}, n)$,
  $H(j, l, n^{\star})$, $H(k^{\star}, l^{\star}, m)$.
\end{definition}

\begin{proposition}
  [TODO: Layout basic equalities for 6j symbols.]
\end{proposition}

\begin{definition}[$10$j-symbol]\label{def/10j-symbol}
  [TODO]
\end{definition}

\begin{definition}[$15$j-symbol]\label{def/15j-symbol}
  [TODO]
\end{definition}

% \subsection{Example}
% tables in appendix?

\section{Topology $(T)$}
\subsection{$4$-manifold}
Manifolds in real dimension four are of interest because their
wildness, witnessed in the following examples:

\begin{enumerate}
  \item Real dimension $4$ is the first dimension in which the
        topological structure, the piecewise-linear structure,
        and the smooth structure cease to coincide.
  \item The euclidean space $\mathbb{R}^{n}$ has exactly one
        smooth structure (up to diffeomorphism) for each
        non-negative integer $n$ except for $(n=4)$, in which
        case (strictly) more than a countably infinite worth of
        inequivalent structures exist \cite{?}.
  \item The Poincare conjecture for the $n$-dimensional sphere
        $S^{n}$ has been resolved except for $n=4$, which remains
        widely open to date \cite{?}.
  \item The physical universe in which we live seems to be
        well-modeled by a $4$-manifold.
\end{enumerate}

\noindent Despite the wildness, Kirby and Siebenmann
\cite{kirby-siebenmann} \cite{turaev-qiok-3-manifolds} show that
the category of smooth manifolds is equivalent to the category of
piecewise-linear manifolds. The data of the later are much
smaller and more elegant. Thus really we will be working on the
piecewise-linear cases, and hence by a manifold we mean a
piecewise-linear, oriented and closed manifold in real dimension
$4$ unless further specified. Readers who are uncomfortable with
the result of Kirby-Siebenmann can treat this entire work as one
about piecewise-linear manifolds instead of the smooth manifolds.

\subsection{Triangulation}

\noindent The standard oriented $4$-simplex is defined to be
following topological subspace (with natural orientation)
$$\Delta = \Delta_{4} = \left\{x \in [0,1]^{5} \,|\, \sum x_{i} = 1 \right\}.$$
By a triangulated ($4$-)space we mean a finite collection of
copies of $\Delta$ with a finite collection of (affine and
oriented) identifications of the pairs of distinct faces. Clearly
an oriented topological space canonically arises from a
triangulated space. We shall confuse both of them. By a
triangulation of a $4$-manifold $X$ we mean a pair of
triangulated space $X'$ and a piecewise-linear homeomorphism
$X \simeq X'$. Two triangulated $4$-spaces give rise to the same
manifold (up to diffeomorphism) if and only if both gluing data
can be relate by a finite sequence of Pachner basic moves
\cite{?}. Thus we can present a smooth $4$-manifold (up to
diffeomorphism) by a gluing datum (up to Pachner moves). As we
will see, the Crane-Yetter state sum transforms the gluing datum
into a number.

\subsection{Handle decomposition}

\noindent By Morse's theory of extremal points, any smooth
manifold admits a handle decomposition. By Cerf theory, two
handle decompositions present the same manifold (up to
diffeomorphism) if and only if both decomposition data are
related by a finite sequence of handle creations, handle
annihilations, and handle slides \cite{?}. A triangulation of a
manifold admits a natural handle decomposition by taking dual.
The correct state sum is the universal state sum \cite{?}; it
transforms the decomposition datum into a number.

\subsection{Shadow}

\noindent A shadow is another type of structure that encodes the
data of a closed $4$-manifold. Roughly speaking, a shadow is a
$2$-polyhedron with extra decorations (called gleams) that
remember the twisting data. A $2$-polyhedron is a topological and
combinatorial object that encodes $3$-dimensional manifolds
\cite{?Matveev}. It is called a pre-foam in the modern literature
\cite{khovanov-robert/foam}.

\begin{definition}[tripod]\label{def/tripod}
  Define the standard tripod to be the topological subspace of
  $\mathbb{R}^{3}$ consisting of the points $(x,y,z)$ such that
  at least two of the entries are zero, and the last entry
  belongs to $[0,1)$. Define a tripod to be any topological space
  homeomorphic to the standard tripod.
\end{definition}

\begin{definition}[cone]\label{def/cone}
  For each topological space $X$, define its standard open cone
  $cone(X)$ to be the quotient space
  $(X \times \mathbb{R_{\geq 0}})/((x,0) \sim (x',0)).$ Define an
  open cone of $X$ to be any topological space homeomorphic to
  $cone(X)$.
\end{definition}

\begin{definition}[local shape]\label{def/local-shape}
  Let $X$ be a topological space and $x \in X$. Denote by $T$ the
  standard tripod and $S$ the $1$-skeleton of the boundary of the
  standard tetrahedron (a trivalent graph with $4$ vertices and
  $6$ edges). Respectively, we say that $x$ is a smooth point, a
  line point, a tetrahedral point, a boundary smooth point, or a
  boundary line point of $X$ if it has a relative neighborhood
  homeomorphic to $(\mathbb{R}^{2},0)$,
  $(T \times \mathbb{R}, (0, 0))$, $(cone(S), (*, 0))$,
  $(\mathbb{R} \times \mathbb{R}_{\geq 0}, (0, 0))$, or
  $(T \times \mathbb{R}_{\geq 0}, (0, 0))$.
\end{definition}

\begin{definition}[simple $2$-polyhedron]\label{def/simple-2-polyhedron}
  A simple $2$-polyhedron with boundary is defined to be a
  piecewise-linear compact CW-complex $P$ of real dimension two,
  such that each of its point $p$ is either a smooth point, a
  line point, a tetrahedral point, a boundary smooth point, or a
  boundary line point. If only the first three types are
  involved, we call $P$ a simple $2$-polyhedron without boundary.
\end{definition}

\begin{definition}[components of a simple $2$-polyhedron]\label{def/components-of-a-simple-2-polyhedron}
  Let $P$ be a simple $2$-polyhedron with boundary. Define the
  set of smooth points (or called interior points) of $P$ to be
  $Int(P)$. Define the set of line points, tetrahedral points,
  and boundary line points to be $sing(P)$. Define the set of
  boundary line points and boundary smooth points to be
  $\partial P$. Call a connected component of $Int(P)$ to be a
  region of $P$; define the set of regions to be $Region(P)$. $P$
  is said to be orientable if each region of $P$ is orientable.
  An orientation of $P$ is an assignment of orientations to each
  of the region.
\end{definition}

\begin{definition}[shadow]\label{def/shadow}
  Let $P$ be a simple $2$-polyhedron, and $A$ an abelian group
  with a distinguished element $\omega \in A$. We define a shadow
  to be a pair of an orientable $2$-polyhedron $P$ and a map
  (called gleam) $gl: Region(P) \to A$. Unless specified further,
  we assume that $A = \mathbb{Z}[\frac{1}{2}]$ and
  $\omega = \frac{1}{2}$.
\end{definition}

\begin{definition}[shadow moves]\label{def/shadow-moves}
  [TODO]
\end{definition}

\begin{definition}[skeleton of a $4$-manifold]\label{def/skeleton-of-a-4-manifold}
  [TODO]
  Let $X$ be a closed $4$-manifold.
\end{definition}

\begin{definition}[shadow of a $4$-manifold]\label{def/shadow-of-a-4-manifold}
  [TODO]
  Let $X$ be a closed $4$-manifold.
\end{definition}

\section{Sum $\left( \int_{T}{A} \right)$}
\subsection{Crane-Yetter state sum}
\subsection{Shadow state sum}
\subsection{Main result: equivalence of both state sums}
