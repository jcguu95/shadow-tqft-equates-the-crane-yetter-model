\section{Introduction}

+ $4$-manifolds are nice and interesting.

+ Want simplicial invariants for $4$-manifolds.

+ There are two state sums.

+ It would be a nice miracle if they differ.

+ We prove that such miracle is just a dream.

\section{Algebra $(A)$}
\subsection{Premodular category}
+ idea
+ data
+ usual notations: $I$, $I^{\star}$, $v_{i}'$..
\subsection{Graphical calculus}
\subsection{$6$j-symbol, $10$j-symbol, and $15$j-symbol}
normalized $6$j symbol..
\subsection{Example}
tables in appendix?

\section{Topology $(T)$}
\subsection{$4$-manifold}
Manifolds in real dimension four are of interest because their
wildness, witnessed in the following examples:

\begin{enumerate}
  \item Real dimension $4$ is the first dimension in which the
        topological structure, the piecewise-linear structure,
        and the smooth structure cease to coincide.
  \item The euclidean space $\mathbb{R}^{n}$ has exactly one
        smooth structure (up to diffeomorphism) for each
        non-negative integer $n$ except for $(n=4)$, in which
        case (strictly) more than a countably infinite worth of
        inequivalent structures exist \cite{?}.
  \item The Poincare conjecture for the $n$-dimensional sphere
        $S^{n}$ has been resolved except for $n=4$, which remains
        widely open to date \cite{?}.
  \item The physical universe in which we live seems to be
        well-modeled by a $4$-manifold.
\end{enumerate}

Despite the wildness, Kirby and Siebenmann
\cite{kirby-siebenmann} \cite{turaev-qiok-3-manifolds} show that
the category of smooth manifolds is equivalent to the category of
piecewise-linear manifolds. The data of the later are much
smaller and more elegant. Thus really we will be working on the
piecewise-linear cases, and hence by a manifold we mean a
piecewise-linear, oriented and closed manifold in real dimension
$4$ unless further specified. Readers who are uncomfortable with
the result of Kirby-Siebenmann can treat this entire work as one
about piecewise-linear manifolds instead of the smooth manifolds.

\subsection{Triangulation}

The standard
oriented $4$-simplex is defined to be following topological
subspace (with natural orientation)
$$\Delta = \Delta_{4} = \left\{x \in [0,1]^{5} \,|\, \sum x_{i} = 1 \right\}.$$
By a triangulated ($4$-)space we mean a finite collection of
copies of $\Delta$ with a finite collection of (affine and
oriented) identifications of the pairs of distinct faces. Clearly
an oriented topological space canonically arises from a
triangulated space. We shall confuse both of them. By a
triangulation of a $4$-manifold $X$ we mean a pair of
triangulated space $X'$ and a piecewise-linear homeomorphism
$X \simeq X'$. Two triangulated $4$-spaces give rise to the same
manifold (up to diffeomorphism) if and only if both gluing data
can be relate by a finite sequence of Pachner basic moves
\cite{?}. Thus we can present a smooth $4$-manifold (up to
diffeomorphism) by a gluing datum (up to Pachner moves). As we
will see, the Crane-Yetter state sum transforms the gluing datum
into a number.

\subsection{Handle decomposition}
By Morse's theory of extremal points, any smooth manifold admits
a handle decomposition. By Cerf theory, two handle decompositions
present the same manifold (up to diffeomorphism) if and only if
both decomposition data are related by a finite sequence of
handle creations, handle annihilations, and handle slides
\cite{?}. A triangulation of a manifold admits a natural handle
decomposition by taking dual. The correct state sum is the
universal state sum \cite{?}; it transforms the decomposition
datum into a number.

\subsection{Shadow}
+ Shadow in general

\section{Sum $\left( \int_{T}{A} \right)$}
\subsection{Crane-Yetter state sum}
\subsection{Shadow state sum}
\subsection{Main result: equivalence of both state sums}
