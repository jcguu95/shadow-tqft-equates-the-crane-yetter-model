\section{Introduction}

Among all other dimensions, $4$D is probably the most wild.

To tackle, tools have been invented and examined.

Of particular importance and interest are QFTs.

+ Importance of QFTs

+++ Quantum invariants for knots and manifolds.

+++ Deep interconnection among representation theory, statistical
physics, and combinatorics.

+++ Topological phase, quantum computing.. etc.

+++ Grand unifying scheme.

+ Usefulness and effectiveness (Donaldson and Seiberg-Witten)

Limitations of current tools and possible strategies (shadow tqft
remains mysterious.)

Sadly, we showed that they are the same.

Merit of this work

+ Eliminate doubts
+ Perhaps shadows are easier to work with in certain cases.
+ Survey on $4$D tqft and state sums

Sections Summary

\section{Algebra $(A)$}
\subsection{Premodular category}
+ idea [..TODO]

+ data

A premodular category is a spherical category that has a braided
structure. In particular, it is semisimple, $\mathbb{k}$-linear,
and fusion. So there the set $I$ of simple objects up to
isomorphism is finite, and one can define a $\mathbb{Z}$-fusion
ring from this set.

+ usual notations: $I$, $I^{\star}$, $v_{i}'$..

We will define some useful notations mostly following
\cite{turaev-qiok-3-manifolds}. For a fixed premodular category
$C$, we often choose a (finite) set of objects
$\{V_{i} | i \in I\}$ so that $V_{i} \in i$ and
$V_{0} = \mathbb{1}$. As taking dual preserves simplicity, for
each $i \in I$ there is a unique element (denoted $i^{\star}$) in
$I$ such that $V_{i^{\star}}$ is isomorphic to $V_{i}^{\star}$;
this also defines an set-involution $\star: I \to I$. We also fix
a set of morphisms (p.313)
$$\{\omega_{i}: V_{i} \to (V_{i^{\star}})^{\star} \,|\, i \in I\}.$$

Define $H^{ijk}$ to be
$Hom_{C}(\mathbb{1}, V_{i} \otimes V_{j} \otimes V_{k})$. (TODO:
need to deal with strictness? Can we really use coherence theorem
to hand-waive this issue?)

Using the spherical structure, we can define for each $i \in I$
the number $dim(i) \in \mathbb{k}$ as the trace of $id_{V_{i}}$
and the number $\nu_{i} \in \mathbb{k}^{\star}$ as the twisting
coefficient (which?) (p.76). We further define
$$\Delta_{C} = \sum_{i \in I} \nu_{i}^{-1}dim(i)^{2}.$$

Choose a number $D = D_{C} \in \mathbb{k}$ such that
$D^{2} = \sum_{i \in I} dim(i)^{2}$.

Further choose for each $i \in I$ a number
$dim'(i) \in \mathbb{k}$ such that $dim'(0) = 1$,
$dim'(i)^{2} = dim(i)$, and $dim'(i^{\star}) = dim'(i)$. Choose
also for each $i \in I$ a number $\nu_{i}'$ such that
$\nu_{0}' = 1$, $()\nu_{i}')^{2} = \nu_{i}$, and
$\nu_{i^{\star}}' = \nu_{i}'$ (p.313).

Notice that the data $C$ and
$(C, \{V_{i}\}, D, \{\omega_{i}\}, dim'(i), \nu'_{i}$ are not
equivalent. However, we will often confuse the later as a
premodular category.

Now, using $\nu_{i}'$, we can define the symmetrized hom spaces
$H(i,j,k)$ (p313 and p314). (TODO: define it). Note that
$H(i,j,k)=H(j,i,k)=\ldots$ are all equal as sets, instead of
merely isomorphic to each other.

\subsection{Graphical calculus}
\subsection{$6$j-symbol, $10$j-symbol, and $15$j-symbol}
normalized $6$j symbol..
\subsection{Example}
tables in appendix?

\section{Topology $(T)$}
\subsection{$4$-manifold}
Manifolds in real dimension four are of interest because their
wildness, witnessed in the following examples:

\begin{enumerate}
  \item Real dimension $4$ is the first dimension in which the
        topological structure, the piecewise-linear structure,
        and the smooth structure cease to coincide.
  \item The euclidean space $\mathbb{R}^{n}$ has exactly one
        smooth structure (up to diffeomorphism) for each
        non-negative integer $n$ except for $(n=4)$, in which
        case (strictly) more than a countably infinite worth of
        inequivalent structures exist \cite{?}.
  \item The Poincare conjecture for the $n$-dimensional sphere
        $S^{n}$ has been resolved except for $n=4$, which remains
        widely open to date \cite{?}.
  \item The physical universe in which we live seems to be
        well-modeled by a $4$-manifold.
\end{enumerate}

Despite the wildness, Kirby and Siebenmann
\cite{kirby-siebenmann} \cite{turaev-qiok-3-manifolds} show that
the category of smooth manifolds is equivalent to the category of
piecewise-linear manifolds. The data of the later are much
smaller and more elegant. Thus really we will be working on the
piecewise-linear cases, and hence by a manifold we mean a
piecewise-linear, oriented and closed manifold in real dimension
$4$ unless further specified. Readers who are uncomfortable with
the result of Kirby-Siebenmann can treat this entire work as one
about piecewise-linear manifolds instead of the smooth manifolds.

\subsection{Triangulation}

The standard
oriented $4$-simplex is defined to be following topological
subspace (with natural orientation)
$$\Delta = \Delta_{4} = \left\{x \in [0,1]^{5} \,|\, \sum x_{i} = 1 \right\}.$$
By a triangulated ($4$-)space we mean a finite collection of
copies of $\Delta$ with a finite collection of (affine and
oriented) identifications of the pairs of distinct faces. Clearly
an oriented topological space canonically arises from a
triangulated space. We shall confuse both of them. By a
triangulation of a $4$-manifold $X$ we mean a pair of
triangulated space $X'$ and a piecewise-linear homeomorphism
$X \simeq X'$. Two triangulated $4$-spaces give rise to the same
manifold (up to diffeomorphism) if and only if both gluing data
can be relate by a finite sequence of Pachner basic moves
\cite{?}. Thus we can present a smooth $4$-manifold (up to
diffeomorphism) by a gluing datum (up to Pachner moves). As we
will see, the Crane-Yetter state sum transforms the gluing datum
into a number.

\subsection{Handle decomposition}
By Morse's theory of extremal points, any smooth manifold admits
a handle decomposition. By Cerf theory, two handle decompositions
present the same manifold (up to diffeomorphism) if and only if
both decomposition data are related by a finite sequence of
handle creations, handle annihilations, and handle slides
\cite{?}. A triangulation of a manifold admits a natural handle
decomposition by taking dual. The correct state sum is the
universal state sum \cite{?}; it transforms the decomposition
datum into a number.

\subsection{Shadow}

Idea:

A shadow is another type of structure that encodes the data of a
closed $4$-manifold. Roughly speaking, a shadow is a
$2$-polyhedron with extra decorations (called gleams) that
remember the twisting data. A $2$-polyhedron is a topological and
combinatorial object that encodes $3$-dimensional manifolds
\cite{?Matveev}. It is called a pre-foam in the modern
literature \cite{khovanov-robert/foam}.

---

(TODO: Do the following again, by first define the $5$ shapes
with numbering. And then define the $2$-p without boundary as a
special case.)

We define a $2$-polyhedron $P$ without boundary as a
piecewise-linear compact CW-complex of real dimension two, such
that each $p \in P$ has a neighborhood that is piecewise-linearly
homeomorphic to either $\mathbb{R}^{2}$, the product of a tripod
and an open interval, or the cone of the $1$-skeleton of a
tetrahedron (TODO: figures). We call $p$ (TODO.. smooth point,
line point,)

A $2$-polyhedron with boundary can be roughly think of the
intersection of $\mathbb{R}^{3}_{x_{1} \geq 0}$ and a
$2$-polyhedron without boundary, where the boundary plane of the
former intersects the later at a generic position (not
intersecting any vertex). More formally, a $2$-polyhedron with
boundary is defined to be a piecewise-linear compact CW-complex
$P$ of real dimension two, such that each $p \in P$ has a
neighborhood PL-homeomorphic to one of the spaces above, to the
half-plane $\mathbb{R}^{2}_{\geq 0}$, or to the product of a
tripod and $[0,1)$.

(TODO: Define $Int(P)$, $sing(X)$, $\partial(X)$. The sing and
partial intersects at the trivalent vertex in partial. And they
union to $P - Int(P)$. Define also region $Region(P)$.)

2. orientation

An orientation of $P$ is an assignment of orientations to each of
the region. $P$ is said to be orientable if each region of $P$ is
orientable.

3. Define shadow as gleamed 2-polyhedra.

Given an abelian group $A$ and a distinguished element
$\omega \in A$, we define a $(A,\omega)$-shadow to be a pair of
an orientable $2$-polyhedron $P$ and a map (called gleam)
$gl: Region(P) \to A$.

4. Describe shadow of $4$-manifolds.

We fix $A$ as the abelian group of the set of half-integers, and
$\omega = \frak{1}{2}$.

(TODO: Given a closed $4$-manifold, define skeleton of $M$.)

(TODO: Then define the gleams.)

\section{Sum $\left( \int_{T}{A} \right)$}
\subsection{Crane-Yetter state sum}
\subsection{Shadow state sum}
\subsection{Main result: equivalence of both state sums}
